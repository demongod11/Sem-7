\documentclass[11]{article}
\usepackage{fullpage}
\usepackage[fleqn]{amsmath}
\usepackage{amsfonts}
\usepackage{amssymb}
\usepackage[pdftex]{graphicx,color}


\begin{document}

\def\RR{\mathbb{R}}
\def\off{\mbox{{\it off}}}
\def\dif{\mbox{{\it dif}}}

\section*{Principles of Cyber-Physical Systems\\
Project: Modeling of Multiple Heaters}


In this project, we will model a room heating system.  Consider a
house with 4 rooms that are heated by 2 heaters.  The temperature in
each room is controlled by a heater, if there is one in it, and also
depends on the temperature of the adjacent rooms and the outside
temperature.  Each room can have at most one heater.  Clearly, at any
given time, only two of the rooms have a heater.

Let $x_i$ be the temperature in room $i$ ($i=1,2,3,4$), $u$ the
outside temperature.  The temperature of a room changes
linearly with the difference of the temperature with the other rooms,
the difference with the outside temperature, and the power of the
heater if there is one in it.  Specifically, the dynamics of the
system is given by
\[ \dot{x}_i = c_i h_i + b_i(u - x_i) + \sum_{j \neq i} a_{i,j} (x_j -
x_i) \] %
with constants $a_{i,j}$, $b_i$, $c_i$.  In this equation, $h_i \in
\{0,1\}$ is the power status of the heater in the room: $h_i$ is $0$
if there is no heater or the heater is off, $h_i = 1$ if there is a
heater and it is on.  We assume that all the heaters are identical.
If $a_{i,j} > 0$ then rooms $i$ and $j$ are adjacent.


\begin{enumerate}
\item Design a Simulink model (in a subsystem block) of the above room
  heating system.  Describe briefly your design in the report.  The
  system should have $h_i$ and $u$ as its inputs.  It should be
  parameterized by the constants $a_{i,j}$ in a matrix $A = [a_{i,j}]
  \in \RR^{4\times 4}$, the constants $b_i$ in a vector $b = [b_i] \in
  \RR^4$, the constants $c_i$ in a vector $c = [c_i] \in \RR^4$, and
  the initial temperature of each room in a vector $x_0 \in \RR^4$.
\item We will control the heaters by a typical thermostat, i.e.\ the
  heater is switched on at its maximum power ($h_i = 1$) if the
  temperature is below a certain threshold, and off ($h_i = 0$) if it
  is beyond another (higher) threshold.  Specifically, for each room
  $i$ we define thresholds $on_i$ and $\off_i$, and the heater in room
  $i$ (if there is one) is on if $x_i \leq on_i$ and off if $x_i \geq
  \off_i$.\\
  We will also control the placement of the heaters in the rooms
  (remember that there are fewer heaters than the rooms) by the
  following rule: a heater is moved from room $j$ to room $i$ ($i \neq
  j$) if all the followings hold:
  \begin{itemize}
  \item room $i$ has no heater (each room can have at most one heater)
  \item room $j$ has a heater
  \item temperature $x_i \leq get_i$
  \item the difference $x_j - x_i \geq \dif_i$.
  \end{itemize}
  The constants $get_i$ and $\dif_i$ may differ for each room.  When a
  heater can be moved to two different rooms or two heaters can be
  moved to the same room, you can make any choice you like (for
  example, always favor the room with higher index, or always favor
  the room with lower temperature).\\
  Design this controller in Stateflow (as a subsystem block),
  parameterized by the constants $on_i$ in a vector $on = [on_i] \in
  \RR^4$, the constants $\off_i$ in a vector $\off = [\off_i] \in
  \RR^4$, the constants $get_i$ in a vector $get = [get_i] \in
  \RR^4$, the constants $\dif_i$ in a vector $\dif = [\dif_i] \in
  \RR^4$.  The inputs to the controller should be the temperatures
  $x_i$.  The outputs should be $h_i$ for each room $i$,
  with the constraints that $h_i \in \{0,1\}$ and at any time, at
  most two of these outputs can be $1$ (because there are only
  two heaters).  Describe rigorously and succinctly your design in the
  report.
\item Connect the controller model to the rooms' model and simulate
  your system with the following data.  The system should be simulated
  for a reasonable amount of time, i.e.\ when interesting events, such
  as moving of heaters, happen.  Plot the temperature $x_i$ of each
  room, the control action $h_i$, and the placement of the heaters.
  \begin{gather*}
    A = \begin{bmatrix}
      0.00 & 0.30 & 0.40 & 0.30 \\
      0.30 & 0.00 & 0.50 & 0.00 \\
      0.40 & 0.50 & 0.00 & 0.30 \\
      0.30 & 0.00 & 0.30 & 0.00
    \end{bmatrix}, \quad
    b = \begin{bmatrix} 0.30 \\ 0.20 \\ 0.50 \\ 0.40 \end{bmatrix},
    \quad
    c = \begin{bmatrix} 9.00 \\ 7.00 \\ 11.00 \\ 7.00 \end{bmatrix}
    \\
    u = 6, \quad
    x_0 = \begin{bmatrix} 16.5 & 16.5 & 16.5 & 16.5 \end{bmatrix}^T
    \\
    \off = \begin{bmatrix} 20 & 20 & 20 & 20 \end{bmatrix}^T, \quad
    on = \begin{bmatrix} 19 & 19 & 19 & 19 \end{bmatrix}^T
    \\
    get = \begin{bmatrix} 17 & 16 & 16 & 17 \end{bmatrix}^T, \quad
    \dif = \begin{bmatrix} 1 & 1 & 1 & 1 \end{bmatrix}^T
    \\
    \text{Initially, the two heaters are in room $2$ and room $3$.}
  \end{gather*}
  We would like to keep the temperature of each room between $15$ and $20$
  degree.  In your simulation, was this requirement satisfied?
\item Change the parameters of the controller: $on_i$, $get_i$,
  $\dif_i$ while $\off_i$ remain unchanged.  Simulate your system and
  discuss the effects of these changes.  Is the temperature
  requirement satisfied in each case?
\end{enumerate}

\end{document}



%%% Local Variables: 
%%% mode: latex
%%% TeX-master: t
%%% End: 
